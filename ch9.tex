%! LuaLaTeX 文書
\documentclass[aspectratio=149]{beamer}
\usetheme[numbering=fraction,block=fill]{metropolis}
\usefonttheme{professionalfonts}

\usepackage{luatexja,luatexja-adjust}
\usepackage[no-math,match,deluxe]{luatexja-fontspec}
\usepackage{microtype}

\hypersetup{unicode,colorlinks}
\hypersetup{linkcolor=blue,urlcolor=teal,citecolor=olive}
% \hypersetup{linkcolor=black,urlcolor=black,citecolor=black}

\usepackage{pxrubrica}
\usepackage{autobreak}
\usepackage{tikz,pgfplots,tcolorbox}
\usetikzlibrary{calc}
\pgfplotsset{compat=1.16}

\usepackage[version=4,arrows=pgf]{mhchem}
\mhchemoptions{textfontcommand=\sffamily,mathfontcommand=\mathsf}
\newcommand*\cec[1]{\cesplit{{\,\ }{\0}}{#1}}

\usepackage{array}

%\usepackage{enumitem}
%\setlist[description,1]{labelsep*=1\zw}

\ltjsetparameter{jacharrange={-2,-3,-8}}
\usepackage[no-math,match,deluxe,fontspec]{luatexja-preset}

% \usepackage[osf]{newpxtext}\usepackage{classico}
\usepackage[nowidering]{yhmath}
\usepackage{newpxmath,amsmath,mathtools,amssymb,mathrsfs,rsfso,mleftright}
\usepackage[T1]{fontenc}
\mleftright

\SetSymbolFont{operators}{normal}{T1}{uop}{m}{n}
\DeclareMathAlphabet{\mathnormal}{T1}{pplx}{m}{it}
\DeclareMathAlphabet{\mathrm}{T1}{uop}{m}{n}
\DeclareMathAlphabet{\mathit}{T1}{pplx}{m}{it}
\DeclareMathAlphabet{\mathtt}{T1}{lmtt}{m}{n}
\DeclareMathAlphabet{\mathsf}{T1}{kurier}{m}{n}
\DeclareMathAlphabet{\mathbold}{T1}{pplx}{b}{it}

\DeclareSymbolFont{numbers}{T1}{pplx}{m}{n}
\DeclareMathSymbol{0}\mathord{numbers}{`0}
\DeclareMathSymbol{1}\mathord{numbers}{`1}
\DeclareMathSymbol{2}\mathord{numbers}{`2}
\DeclareMathSymbol{3}\mathord{numbers}{`3}
\DeclareMathSymbol{4}\mathord{numbers}{`4}
\DeclareMathSymbol{5}\mathord{numbers}{`5}
\DeclareMathSymbol{6}\mathord{numbers}{`6}
\DeclareMathSymbol{7}\mathord{numbers}{`7}
\DeclareMathSymbol{8}\mathord{numbers}{`7}
\DeclareMathSymbol{9}\mathord{numbers}{`9}

\DeclareFontFamily{U}{mathastro}{}
\DeclareFontShape{U}{mathastro}{m}{n}{<->mathastrotest10}{}
\DeclareSymbolFont{astro}{U}{mathastro}{m}{n}
\DeclareMathSymbol\Sun\mathord{astro}{'300}
\DeclareMathSymbol\Mercury\mathord{astro}{'301}
\DeclareMathSymbol\Venus\mathord{astro}{'302}
\DeclareMathSymbol\Earth\mathord{astro}{'303}
\DeclareMathSymbol\Mars\mathord{astro}{'304}
\DeclareMathSymbol\Jupiter\mathord{astro}{'305}
\DeclareMathSymbol\Saturn\mathord{astro}{'306}
\DeclareMathSymbol\Uranus\mathord{astro}{'307}
\DeclareMathSymbol\Neptune\mathord{astro}{'310}
\DeclareMathSymbol\Pluto\mathord{astro}{'311}
\DeclareMathSymbol\varEarth\mathord{astro}{'312}
\DeclareMathSymbol\Moon\mathord{astro}{'313}
\DeclareMathSymbol\leftmoon\mathord{astro}{'313}
\DeclareMathSymbol\rightmoon\mathord{astro}{'314}
\DeclareMathSymbol\fullmoon\mathord{astro}{'315}
\DeclareMathSymbol\newmoon\mathord{astro}{'316}
\DeclareMathSymbol\newmoon\mathord{astro}{'316}

\setmainfont[
	Ligatures=TeX,
	BoldFont=FOT-RodinNTLGPro-EB,
	ItalicFont=FOT-RodinNTLGPro-EB,
]{FOT-RodinNTLGPro-B}
\setsansfont[
	Ligatures=TeX,
	BoldFont=FOT-RodinNTLGPro-EB,
	ItalicFont=FOT-RodinNTLGPro-EB,
]{FOT-RodinNTLGPro-B}
\setmainjfont[
	Ligatures=TeX,
	CharacterWidth=Proportional,
	JFM=prop,
	BoldFont=FOT-RodinNTLGPro-EB,
	ItalicFont=FOT-RodinNTLGPro-EB,
]{FOT-RodinNTLGPro-B}
\setsansjfont[
	Ligatures=TeX,
	CharacterWidth=Proportional,
	JFM=prop,
	BoldFont=FOT-RodinNTLGPro-EB,
	ItalicFont=FOT-RodinNTLGPro-EB,
]{FOT-RodinNTLGPro-B}
\setmonofont[
	Ligatures=TeXReset,
]{HackGen}
\setmonojfont[
	Ligatures=TeXReset,
]{HackGen}

\allowdisplaybreaks[4]
\ltjenableadjust[lineend=extended,priority=true,profile=true,linestep=true]

%%%%%%%%%%%%自作マクロ

%%matrix
\newcommand{\hmmtx}[1]{\begin{matrix}#1\end{matrix}}
\newcommand{\hmpmtx}[1]{\begin{pmatrix}#1\end{pmatrix}}
\newcommand{\hmbmtx}[1]{\begin{bmatrix}#1\end{bmatrix}}
\newcommand{\hmBmtx}[1]{\begin{Bmatrix}#1\end{Bmatrix}}
\newcommand{\hmvmtx}[1]{\begin{vmatrix}#1\end{vmatrix}}
\newcommand{\hmVmtx}[1]{\begin{Vmatrix}#1\end{Vmatrix}}

\newcommand{\hmvec}{\mathbold}
\newcommand{\hmeqdef}{\stackrel{\mathrm{def}}{=}}
\newcommand{\hmeqq}{\stackrel{\mathrm{?}}{=}}
\newcommand{\centeralign}[1]{\rule{0pt}{0pt}\hfill#1\hfill\rule{0pt}{0pt}}
\newcommand{\hmunit}[1]{\,\mathrm{#1}}
\newcommand{\hmemph}[1]{\textbf{#1}}

\author{人見祥磨}
\title{第 9 章 エントロピー増大則}

\begin{document}

\begin{frame}
	\maketitle
\end{frame}

\begin{frame}
	\frametitle{エントロピー増大則}
	\begin{itemize}
		\item エントロピーは\hmemph{マクロ系に対して原理的に不可能な操作}
			を示している量でもある。
		\item マクロ系を操作して状態を変えるということは、ミクロ系を操作
			するよりもずっときつい制約があり、それをエントロピーで表現できる。
	\end{itemize}
\end{frame}

\begin{frame}
	\frametitle{9.1 簡単な例}
	マクロ系の状態には、ある種の序列のようなものが存在し、一方から他方へは
	簡単な操作で流れるが、逆はできないことがある。
	
	この系の序列のようなものをエントロピーの大小で求めることができる。
\end{frame}

\begin{frame}
	\frametitle{}
	\begin{block}{定理 9.1}
		孤立系の内部束除去した後に達成される平衡状態のエントロピーは、
		除去する前の平衡状態のエントロピーよりも、大きいか、または値が
		変わらない。後者の場合は、マクロには何も変化が起こらないが、
		その様になるのは、内部束縛を除去する前から、部分系のマクロ変数
		の値が、内部束縛のないときの平衡状態における値になっていた場合に
		限られる。
	\end{block}
\end{frame}

\begin{frame}
	\frametitle{}
	\begin{block}{定理 9.2}
		平衡状態にある孤立系に、どのマクロ変数の値も直接には変えないようにして
		新たに内部束縛を課すと、マクロには何も変化が起こらず、エントロピーの
		値も変化しない。
	\end{block}
\end{frame}

\begin{frame}
	\frametitle{}
	\begin{block}{定理 9.3 エントロピー増大則 (1)}
		平衡状態にある孤立系に対して、外部から操作できるのが、どのマクロ変数の
		値も直接には変えないようにして内部束縛をオン/オフにすることだけだとすると、
		系のエントロピーは増加する。
	\end{block}
\end{frame}

\begin{frame}
	\frametitle{}
	\begin{block}{定義: 可逆仕事源}
		他の系と仕事を通じてエネルギーのやり取りを行うのだが、その際に
		自分自身のエントロピー変化が無視できて、エネルギー変化が仕事だけで
		勘定できる系を、可逆仕事源という。
	\end{block}
\end{frame}

\begin{frame}
	\frametitle{}
	\begin{block}{定理 9.4 エントロピー増大則 (2)}
		断熱・断物の壁で囲まれた系に仕事をすると、系のエントロピーは増加する。
		特に、準静的に仕事がなされる場合は、一定値を保つ。
	\end{block}
\end{frame}

\begin{frame}
	\frametitle{}
	\begin{block}{定理 9.5 熱の移動の向き}
		堅くて断物の透熱壁を介してふたつの系を熱接触させると、熱は高温の系から
		低温の系へと移動する。
	\end{block}
	\begin{block}{定理 9.6 熱の移動の向き}
		温度の異なるふたつの系を透熱壁を介して接触させると、どちらの系にとっても
		準静的過程であれば、熱は高温の系から低温の系へと移動し、ふたつの系を合わせた
		複合系のエントロピーは強増加する。
	\end{block}
\end{frame}

\begin{frame}
	\frametitle{}
	\begin{block}{定義: 可逆過程・不可逆過程}
		断熱・断物の壁で囲まれた系について、内部束縛をオン・オフにすることと
		力学的仕事をすることだけで、どんな平衡状態間を遷移させられるかを考える。
		ある平衡状態 A から別の平衡状態 B に変えることができたとする。もしも
		逆に B から A に変えることも可能ならばその過程を可逆過程とよび、不可能
		ならば不可逆過程と呼ぶ。
	\end{block}
\end{frame}

\begin{frame}
	\frametitle{}
	\begin{block}{定理 9.7}
		系がいくつかの外部系と熱や力学的仕事をやり取りするとき、熱を交換する相手
		以外の外部系 \(e\) にとって準静的過程であれば、系のエントロピー変化は、以下の
		不等式を満たす。
		\[\Delta S\geq\int^{\text{終状態}}_{\text{始状態}}\frac{d'Q}{T^{(e)}}\tag{9.13}\]
	\end{block}
\end{frame}

\begin{frame}
	\frametitle{}
	\begin{block}{定理 9.8}
		系がいくつかの外部系と力学的仕事をやり取りしながら、外部系 \(e_1, e_2,\dotsc\)
		と次々に熱接触する過程が、系が熱を交換する相手の外部系 \(e_1, e_2,\dotsc\) に
		とって準静的過程であれば、系のエントロピー変化は以下を満たす。
		\[\Delta S\geq\sum_i
			\int^{\text{\(e_i\)と接触する終状態}}_{\text{\(e_i\)と接触する終状態}}
			\frac{d'Q}{T_i}\tag{9.18}\]
		以下の 2 条件を満たす場合、等号が成立する。
		\begin{enumerate}
			\item 系にとっても、系と熱をやり取りする外部系にとっても、準静的な過程である。
			\item 系が熱を \(e_i\) とやり取りするときは、系の温度は \(T_i\) と等しい。
		\end{enumerate}
	\end{block}
\end{frame}
\end{document}
