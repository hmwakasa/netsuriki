% +++
% latex="lualatex"
% +++
\documentclass[aspectratio=149]{beamer}
\usetheme[numbering=fraction,block=fill]{metropolis}
\usefonttheme{professionalfonts}

\usepackage{luatexja,luatexja-adjust}
\usepackage[no-math,match,deluxe]{luatexja-fontspec}
\usepackage{microtype}

\hypersetup{unicode,colorlinks}
\hypersetup{linkcolor=blue,urlcolor=teal,citecolor=olive}
% \hypersetup{linkcolor=black,urlcolor=black,citecolor=black}

\usepackage{pxrubrica}
\usepackage{autobreak}
\usepackage{tikz,pgfplots,tcolorbox}
\usetikzlibrary{calc}
\pgfplotsset{compat=1.16}

\usepackage[version=4,arrows=pgf]{mhchem}
\mhchemoptions{textfontcommand=\sffamily,mathfontcommand=\mathsf}
\newcommand*\cec[1]{\cesplit{{\,\ }{\0}}{#1}}

\usepackage{array}

%\usepackage{enumitem}
%\setlist[description,1]{labelsep*=1\zw}

\ltjsetparameter{jacharrange={-2,-3,-8}}
\usepackage[no-math,match,deluxe,fontspec]{luatexja-preset}

% \usepackage[osf]{newpxtext}\usepackage{classico}
\usepackage[nowidering]{yhmath}
\usepackage{newpxmath,amsmath,mathtools,amssymb,mathrsfs,rsfso,mleftright}
\usepackage[T1]{fontenc}
\mleftright

\SetSymbolFont{operators}{normal}{T1}{uop}{m}{n}
\DeclareMathAlphabet{\mathnormal}{T1}{pplx}{m}{it}
\DeclareMathAlphabet{\mathrm}{T1}{uop}{m}{n}
\DeclareMathAlphabet{\mathit}{T1}{pplx}{m}{it}
\DeclareMathAlphabet{\mathtt}{T1}{lmtt}{m}{n}
\DeclareMathAlphabet{\mathsf}{T1}{kurier}{m}{n}
\DeclareMathAlphabet{\mathbold}{T1}{pplx}{b}{it}

\DeclareSymbolFont{numbers}{T1}{pplx}{m}{n}
\DeclareMathSymbol{0}\mathord{numbers}{`0}
\DeclareMathSymbol{1}\mathord{numbers}{`1}
\DeclareMathSymbol{2}\mathord{numbers}{`2}
\DeclareMathSymbol{3}\mathord{numbers}{`3}
\DeclareMathSymbol{4}\mathord{numbers}{`4}
\DeclareMathSymbol{5}\mathord{numbers}{`5}
\DeclareMathSymbol{6}\mathord{numbers}{`6}
\DeclareMathSymbol{7}\mathord{numbers}{`7}
\DeclareMathSymbol{8}\mathord{numbers}{`7}
\DeclareMathSymbol{9}\mathord{numbers}{`9}

\DeclareFontFamily{U}{mathastro}{}
\DeclareFontShape{U}{mathastro}{m}{n}{<->mathastrotest10}{}
\DeclareSymbolFont{astro}{U}{mathastro}{m}{n}
\DeclareMathSymbol\Sun\mathord{astro}{'300}
\DeclareMathSymbol\Mercury\mathord{astro}{'301}
\DeclareMathSymbol\Venus\mathord{astro}{'302}
\DeclareMathSymbol\Earth\mathord{astro}{'303}
\DeclareMathSymbol\Mars\mathord{astro}{'304}
\DeclareMathSymbol\Jupiter\mathord{astro}{'305}
\DeclareMathSymbol\Saturn\mathord{astro}{'306}
\DeclareMathSymbol\Uranus\mathord{astro}{'307}
\DeclareMathSymbol\Neptune\mathord{astro}{'310}
\DeclareMathSymbol\Pluto\mathord{astro}{'311}
\DeclareMathSymbol\varEarth\mathord{astro}{'312}
\DeclareMathSymbol\Moon\mathord{astro}{'313}
\DeclareMathSymbol\leftmoon\mathord{astro}{'313}
\DeclareMathSymbol\rightmoon\mathord{astro}{'314}
\DeclareMathSymbol\fullmoon\mathord{astro}{'315}
\DeclareMathSymbol\newmoon\mathord{astro}{'316}
\DeclareMathSymbol\newmoon\mathord{astro}{'316}

\setmainfont[
	Ligatures=TeX,
	BoldFont=FOT-RodinNTLGPro-EB,
	ItalicFont=FOT-RodinNTLGPro-EB,
]{FOT-RodinNTLGPro-B}
\setsansfont[
	Ligatures=TeX,
	BoldFont=FOT-RodinNTLGPro-EB,
	ItalicFont=FOT-RodinNTLGPro-EB,
]{FOT-RodinNTLGPro-B}
\setmainjfont[
	Ligatures=TeX,
	CharacterWidth=Proportional,
	JFM=prop,
	BoldFont=FOT-RodinNTLGPro-EB,
	ItalicFont=FOT-RodinNTLGPro-EB,
]{FOT-RodinNTLGPro-B}
\setsansjfont[
	Ligatures=TeX,
	CharacterWidth=Proportional,
	JFM=prop,
	BoldFont=FOT-RodinNTLGPro-EB,
	ItalicFont=FOT-RodinNTLGPro-EB,
]{FOT-RodinNTLGPro-B}
\setmonofont[
	Ligatures=TeXReset,
]{HackGen}
\setmonojfont[
	Ligatures=TeXReset,
]{HackGen}

\allowdisplaybreaks[4]
\ltjenableadjust[lineend=extended,priority=true,profile=true,linestep=true]

%%%%%%%%%%%%自作マクロ

%%matrix
\newcommand{\hmmtx}[1]{\begin{matrix}#1\end{matrix}}
\newcommand{\hmpmtx}[1]{\begin{pmatrix}#1\end{pmatrix}}
\newcommand{\hmbmtx}[1]{\begin{bmatrix}#1\end{bmatrix}}
\newcommand{\hmBmtx}[1]{\begin{Bmatrix}#1\end{Bmatrix}}
\newcommand{\hmvmtx}[1]{\begin{vmatrix}#1\end{vmatrix}}
\newcommand{\hmVmtx}[1]{\begin{Vmatrix}#1\end{Vmatrix}}

\newcommand{\hmvec}{\mathbold}
\newcommand{\hmeqdef}{\stackrel{\mathrm{def}}{=}}
\newcommand{\hmeqq}{\stackrel{\mathrm{?}}{=}}
\newcommand{\centeralign}[1]{\rule{0pt}{0pt}\hfill#1\hfill\rule{0pt}{0pt}}
\newcommand{\hmunit}[1]{\,\mathrm{#1}}
\newcommand{\hmemph}[1]{\textbf{#1}}

\author{人見祥磨}
\title{第 9 章 エントロピー増大則}

\begin{document}

\begin{frame}
	\maketitle
\end{frame}

\begin{frame}
	\frametitle{エントロピー増大則}
	\begin{itemize}
		\item エントロピーは\hmemph{マクロ系に対して原理的に不可能な操作}
			を示している量でもある。
		\item マクロ系を操作して状態を変えるということは、ミクロ系を操作
			するよりもずっときつい制約があり、それをエントロピーで表現できる。
	\end{itemize}
\end{frame}

\begin{frame}
	\frametitle{9.1 簡単な例}
	マクロ系の状態には、ある種の序列のようなものが存在し、一方から他方へは
	簡単な操作で流れるが、逆はできないことがある。
	
	この系の序列のようなものをエントロピーの大小で求めることができる。
\end{frame}

\begin{frame}
	\frametitle{9.2.1 内部束縛条件のオン・オフ}
	十分に大きい容器を考えれば、十分な精度で、コックの開閉がマクロ変数を
	直接変えることはないとみなせる。すなわち、コックの開閉はマクロ変数を
	変えないまま、内部束縛条件だけを変えるとみなすことができる。
	
	しかし、内部束縛条件を変えるだけで気体が自発的に動き始める。
\end{frame}

\begin{frame}
	\frametitle{9.2.2 内部束縛条件のオン・オフに伴うエントロピー変化}
	\begin{block}{定理 3.2}
		任意の内部束縛 \(C_k\) について、それがあるときのエントロピーの値は、
		ないときのエントロピーの値以下である。
		\begin{multline}
			S[U,X_1,\dots X_t;\dots,C_{k-1},C_k,C_{k+1},\dotsc]\\
			\leq S[U,X_1,\dots X_t;\dots,C_{k-1},C_{k+1},\dotsc]\tag{3.38}
		\end{multline}
	\end{block}
\end{frame}

\begin{frame}
	\frametitle{9.2.2 内部束縛条件のオン・オフに伴うエントロピー変化}
	定理 3.2 は次のように言い換えることができる。
	\begin{block}{定理 9.1}
		孤立系の内部束除去した後に達成される平衡状態のエントロピーは、
		除去する前の平衡状態のエントロピーよりも、大きいか、または値が
		変わらない。後者の場合は、マクロには何も変化が起こらないが、
		その様になるのは、内部束縛を除去する前から、部分系のマクロ変数
		の値が、内部束縛のないときの平衡状態における値になっていた場合に
		限られる。
	\end{block}
\end{frame}

\begin{frame}
	\frametitle{9.2.2 内部束縛条件のオン・オフに伴うエントロピー変化}
	内部条件を課す場合を考える。
	\[
		S[U,X_1,\dots X_t;C_1,\dots,C_b]
		\geq \sum_i S^{(i)}[U^{(i)},X^{(i)}_1,\dots X^{(i)}_{t_i}]\tag{3.35}
	\]
	すでに式 (3.35) の右辺は最大値なので、変化しない。
	
	\begin{block}{定理 9.2}
		平衡状態にある孤立系に、どのマクロ変数の値も直接には変えないようにして
		新たに内部束縛を課すと、マクロには何も変化が起こらず、エントロピーの
		値も変化しない。
	\end{block}
\end{frame}

\begin{frame}
	\frametitle{9.2.3 孤立系のエントロピー増大則}
	定理 9.1 と 9.2 から次を得る。
	\begin{block}{定理 9.3 エントロピー増大則 (1)}
		平衡状態にある孤立系に対して、外部から操作できるのが、どのマクロ変数の
		値も直接には変えないようにして内部束縛をオン/オフにすることだけだとすると、
		系のエントロピーは増加する。
	\end{block}
	
	単に、\hmemph{孤立系のエントロピーは増大する}とも表現し、
	\[\Delta S\geq0\quad\text{(孤立系)}\tag{9.3}\]
	などと書くこともある。
	
	これは、\hmemph{熱力学第 2 法則}のひとつの表現である。
\end{frame}

\begin{frame}
	\frametitle{9.3 部分系のエントロピーの変化}
	\begin{itemize}
		\item 孤立系も、孤立系を含む十分大きな系(全系)を考えれば、孤立系とみなせる。
		\item 全系は孤立系なので、(全系のエントロピーが増大するよう)他の部分系の
			エントロピーを増大させ、特定の部分系のエントロピーを減少させることは
			可能である。
		\item しかしながら、全く無制限に部分系のエントロピーを減少させることは、実は
			できない。
			\begin{itemize}
				\item \(\Delta S_\text{全系}\geq0\) を満たしても、
					\(\Delta S^{(1)}\geq0\) が禁止されるケースがある。
			\end{itemize}
	\end{itemize}
\end{frame}

\begin{frame}
	\frametitle{9.3.1 可逆仕事限}
	\begin{block}{定義: 可逆仕事源}
		他の系と仕事を通じてエネルギーのやり取りを行うのだが、その際に
		自分自身のエントロピー変化が無視できて、エネルギー変化が仕事だけで
		勘定できる系を、可逆仕事源という。
	\end{block}
	例えば、
	\begin{itemize}
		\item 重りをゆっくりと上下させて、全体の位置エネルギーだけを変化させるような系。
			\(d'Q=0=T\,dS\) で、エントロピー変化なし。
		\item 断熱可動壁で気体 A, B が区切られていて、圧力差で可動壁が動くとする。
			片方の気体 A は平衡状態に緩和する暇がなく非平衡状態になり、もう片方 B は
			準静的過程になっているとする。B を断熱壁で区切れば、\(0=d'Q=T\,dS\) で
			エントロピー変化なし。
	\end{itemize}
\end{frame}

\begin{frame}
	\frametitle{9.4 断熱された系のエントロピー増大則}
	可逆仕事源が部分系に仕事をするときのエントロピーの変化は
	\[\Delta S_\text{全系}=\Delta S_\text{部分系}+\Delta S_\text{可逆仕事源}\geq0\]
	であるから、
	\[\Delta S_\text{部分系}\geq0\tag{9.4}\]
	となる。ここで、部分系は可逆仕事源を介して仕事を受け取るだけなので、可逆仕事源
	でない仕事源から全く同じ仕事を受け取ることもできるし、どちらでも部分系にとっては
	全く同じで、(9.4) 式が成り立つ。
	\begin{block}{定理 9.4 エントロピー増大則 (2)}
		断熱・断物の壁で囲まれた系に仕事をすると、系のエントロピーは増加する。
		特に、準静的に仕事がなされる場合は、一定値を保つ。
	\end{block}
\end{frame}

\begin{frame}
	\frametitle{9.4 断熱された系のエントロピー増大則}
	定理 9.4 もエントロピー増大則だとか、熱力学第 2 法則と呼ばれる。言い換えれば、
	\begin{itemize}
		\item 断熱・断物の壁で囲まれた系は、どんな力学的仕事をしようとも、
			必ず系のエントロピーが増加する。
		\item 一旦系のエントロピーが増加すると、力学的仕事だけで元の状態には戻らない。
		\item 系のエントロピーを減少させるには、熱や物質の出入りを許すしかない。
	\end{itemize}
\end{frame}

\begin{frame}
	\frametitle{9.4 熱の移動の向き}
	ふたつの系 \(H, L\) がエネルギー \(U_H, U_L\) で平衡状態になっていて、
	それらを堅くて断物の透熱壁を介して熱接触させる。最初は逆温度 \(B\) が異なっていたとする。
	\[B_H[U_H]\neq B_L[U_L]\tag{9.5}\]
	平衡状態では、熱 \(Q\neq0\) をやり取りして \(B\) が等しくなる。
	\[B_H[U_H-Q]=B_L[U_L+Q]\tag{9.6}\]
	\(Q>0\) だとすると、\(B\) は \(U\) に関して単調で、
	\[B_H[U_H-Q]\geq B_H[U_H]\quad B_L[U_L+Q]\leq B_L[U_L]\tag{9.7}\]
	以上の式と、温度 \(T=1/B\) より、\(T_H[U_H]\geq T_L[U_L]\) を得る。
\end{frame}

\begin{frame}
	\frametitle{9.4 熱の移動の向き}
	\begin{block}{定理 9.5 熱の移動の向き}
		堅くて断物の透熱壁を介してふたつの系を熱接触させると、熱は高温の系から
		低温の系へと移動する。
	\end{block}
	エントロピー増大則より、\(H, L\) をあわせた全系のエントロピーは増大しているので、
	熱が流れるのはエントロピー増大則の帰結と見ることもできる。
	
	これは途中が非平衡状態でも成り立つ。
\end{frame}

\begin{frame}
	\frametitle{9.4 熱の移動の向き}
	準静的な高温系 \(H\) と低温系 \(L\) を熱接触させ、わずかの熱 \(d'Q\neq0\) が
	\(H\) から \(L\) へ流れたところで断熱すと、断熱の前後は平衡状態で、
	\[d'Q=-T_H\,dS_H=T_L\,dS_L\tag{9.9}\]
	となり、
	\[dS_\text{全系}=dS_H+dS_L=\left(\frac{1}{T_L}-\frac{1}{T_H}\right)d'Q\tag{9.10}\]
	となるが、\(dS_\text{全系}\geq0,T_H>T_L\) より、\(d'Q>0,dS_\text{全系}>0\) を得る。
	\begin{block}{定理 9.6 熱の移動の向き}
		温度の異なるふたつの系を透熱壁を介して接触させると、どちらの系にとっても
		準静的過程であれば、熱は高温の系から低温の系へと移動し、ふたつの系を合わせた
		複合系のエントロピーは強増加する。
	\end{block}
\end{frame}

\begin{frame}
	\frametitle{9.5 可逆過程と不可逆過程}
	\begin{block}{定義: 可逆過程・不可逆過程}
		断熱・断物の壁で囲まれた系について、内部束縛をオン・オフにすることと
		力学的仕事をすることだけで、どんな平衡状態間を遷移させられるかを考える。
		ある平衡状態 A から別の平衡状態 B に変えることができたとする。もしも
		逆に B から A に変えることも可能ならばその過程を可逆過程とよび、不可能
		ならば不可逆過程と呼ぶ。
	\end{block}
	内部束縛のオン・オフではエントロピーを下げられないため、不可逆過程が存在する。
	
	これはミクロ系にはない、マクロ系に特有な制限である。
\end{frame}

\begin{frame}
	\frametitle{9.6 熱とエントロピー}
	着目系(系)が外部系 \(e\) と透熱壁と接触していて、断熱ピストンで外部系 \(e'\)
	と力学的仕事をやり取りすることができるとする。最初平衡状態にあった系が、
	あらたな平衡状態になったとして、系のエントロピーが \(\Delta S\) 変化し、
	全部で \(Q\) だけの熱が系に流れ込んだとする。\(\Delta S\) と \(Q\) の関係を考察する。
	
	(6.31) 式より
	\[\Delta S^{(e)}=-\int^{\text{終状態}}_{\text{始状態}}\frac{d'Q}{T^{(e)}}\tag{9.11}\]
	\(\Delta S_\text{全系}=\Delta S+\Delta S^{(e)}+\Delta S^{(e')}\geq0\)なので、
	\[\Delta S\geq\int^{\text{終状態}}_{\text{始状態}}\frac{d'Q}{T^{(e)}}-\Delta S^{(e')}\tag{9.12}\]
\end{frame}

\begin{frame}
	\frametitle{9.6 熱とエントロピー}
	可逆仕事源を考えたときと同じ議論で、\(\Delta S^{(e')}\) であるから、
	\begin{block}{定理 9.7}
		系がいくつかの外部系と熱や力学的仕事をやり取りするとき、熱を交換する相手
		以外の外部系 \(e\) にとって準静的過程であれば、系のエントロピー変化は、以下の
		不等式を満たす。
		\[\Delta S\geq\int^{\text{終状態}}_{\text{始状態}}\frac{d'Q}{T^{(e)}}\tag{9.13}\]
	\end{block}
\end{frame}

\begin{frame}
	\frametitle{9.6 熱とエントロピー}
	つぎつぎと系が異なる外部系と熱接触する場合でも、定理 9.7 を繰り返し適用すれば、
	\begin{block}{定理 9.8}
		系がいくつかの外部系と力学的仕事をやり取りしながら、外部系 \(e_1, e_2,\dotsc\)
		と次々に熱接触する過程が、系が熱を交換する相手の外部系 \(e_1, e_2,\dotsc\) に
		とって準静的過程であれば、系のエントロピー変化は以下を満たす。
		\[\Delta S\geq\sum_i
		\int^{\text{\(e_i\)と接触する終状態}}_{\text{\(e_i\)と接触する始状態}}
		\frac{d'Q}{T_i}\tag{9.18}\]
	\end{block}
\end{frame}

\begin{frame}
	\frametitle{9.6 熱とエントロピー}
	\begin{block}{定理 9.8 (続き)}
		以下の 2 条件を満たす場合、等号が成立する。
		\begin{enumerate}
			\item 系にとっても、系と熱をやり取りする外部系にとっても、準静的な過程である。
			\item 系が熱を \(e_i\) とやり取りするときは、系の温度は \(T_i\) と等しい。
		\end{enumerate}
	\end{block}
	条件 1 を満たせば、定理 6.5 が使えて、条件 2 を満たせば、 \(T=T^{(e)}\) だから、
	\[\Delta S=\int^{\text{終状態}}_{\text{始状態}}\frac{d'Q}{T}
	=\int^{\text{終状態}}_{\text{始状態}}\frac{d'Q}{T^{(e)}}\tag{9.17}\]
\end{frame}
\end{document}
